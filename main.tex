\documentclass[aspectratio=169,10pt]{beamer}
\usepackage[utf8]{inputenc}
\usepackage[ngerman]{babel}
\usepackage{csquotes}
\usepackage{graphicx}
\usepackage{booktabs}
\usepackage{hyperref}
\usepackage{siunitx}

\usetheme{HM} % our custom theme

% --- Meta ---
\title[Geschäftsprozessmanagement]{Geschäftsprozessmanagement}
\subtitle{Organisation}
\author[Prof. Dr. Marcus Fischer]{Prof. Dr. Marcus Fischer}
\institute{Hochschule München\\University of Applied Sciences\\Fakultät für Mathematik und Informatik\\Professur für Enterprise Architecture Management und betriebliche Informationssysteme}
\date[Sommersemester 2025]{Sommersemester 2025}

% Left column on the cover slide
\sethmcoverleft{\insertinstitute\\[0.75em]\insertauthor}

% Optional additional footer information (appears after title and author)
% \sethmfooter{Weiterführende Informationen}

% If your logo has a different filename, set it here:
% \sethmlogo{hm_logo.pdf}

\begin{document}

\begin{frame}[plain]
  \titlepage
\end{frame}

\begin{frame}{Inhalt}
  \tableofcontents
\end{frame}

\section{Einführung}
\begin{frame}{Einführung}{Überblick}
\hmhighlight{Worum geht es?}
\begin{itemize}
  \item Stellen Sie hier die wichtigsten Punkte Ihrer Präsentation vor.
  \item Verwenden Sie kurze, prägnante Aussagen und ergänzen Sie sie ggf. um Stichworte.
  \item Betonen Sie zentrale Kennzahlen oder Ergebnisse mit \textbf{fetter Schrift}.
\end{itemize}
\end{frame}

\section{Beispielblöcke}
\begin{frame}{Beispielblöcke}{Strukturierte Inhalte}
\hmhighlight{Designrichtlinien}
\begin{block}{Hinweis}
  Neutrale graue Kästen für strukturierte Inhalte.
\end{block}
\begin{exampleblock}{Beispiel}
  Praxisbeispiel – kurze, klare Aussage.
\end{exampleblock}
\end{frame}

\section{Zusammenfassung}
\begin{frame}{Zusammenfassung}{Key Takeaways}
\hmhighlight{Das sollten Sie mitnehmen}
\begin{itemize}
  \item Fassen Sie die Kernaussagen zusammen und leiten Sie zu nächsten Schritten über.
  \item Nutzen Sie optional die Akzentfarbe \textcolor{HMRed}{HM-Rot} für Highlights.
  \item Schließen Sie mit einem klaren Call-to-Action oder einer Kontaktinformation.
\end{itemize}
\end{frame}

\end{document}
