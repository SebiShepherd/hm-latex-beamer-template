\documentclass[aspectratio=169,10pt]{beamer}
\usepackage[utf8]{inputenc}
\usepackage[ngerman]{babel}
\usepackage{csquotes}
\usepackage{graphicx}
\usepackage{booktabs}
\usepackage{hyperref}
\usepackage{siunitx}

\usetheme{HM} % our custom theme

% --- Meta ---
\title[Geschäftsprozessmanagement]{Geschäftsprozessmanagement}
\subtitle{Organisation}
\author[Prof. Dr. Marcus Fischer]{Prof. Dr. Marcus Fischer}
\institute{Hochschule München \\ University of Applied Sciences \\ Fakultät für Mathematik und Informatik \\ Professur für Enterprise Architecture Management und betriebliche Informationssysteme}
\date{Sommersemester 2025}

% If your logo has a different filename, set it here:
% \sethmlogo{hm_logo.pdf}

\begin{document}

\begin{frame}[plain]
  \titlepage
\end{frame}

\begin{frame}{Geschäftsprozessmanagement}{Organisatorisches}
  \hmredbox{Die Prüfung}
  \begin{itemize}
    \item Die Prüfung ist schriftlich und geht über 90 Minuten.
    \item Es sind alle Inhalte der Vorlesungen, Übungen und Praxisvorträge prüfungsrelevant.
    \item In der Prüfung können insgesamt 90 Punkte erreicht werden.
    \item In der letzten Vorlesungswoche findet in der Übung eine Probeklausur mit anschließender Besprechung statt.
  \end{itemize}
\end{frame}

\begin{frame}{Modulelemente}{Überblick}
  \hmredbox{Ablauf}
  \begin{columns}[T,onlytextwidth]
    \column{0.48\textwidth}
    \begin{itemize}
      \item Vorlesung mit Fokusthemen aus dem Prozessmanagement.
      \item Übungen mit Werkzeugunterstützung und Fallstudien.
    \end{itemize}
    \column{0.48\textwidth}
    \begin{itemize}
      \item Praxisvorträge von Unternehmen aus der Region.
      \item Leistungsnachweis in Form von Klausur und Projektarbeit.
    \end{itemize}
  \end{columns}
\end{frame}

\begin{frame}{Kontakt}{Fragen}
  \hmredbox{Sprechstunden}
  \begin{itemize}
    \item Persönliche Termine nach Vereinbarung per E-Mail.
    \item Zusätzliche Materialien finden Sie im Moodle-Kursraum.
  \end{itemize}
\end{frame}

\end{document}
