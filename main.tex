\documentclass[aspectratio=169,10pt]{beamer}
\usepackage[utf8]{inputenc}
\usepackage[ngerman]{babel}
\usepackage{csquotes}
\usepackage{graphicx}
\usepackage{booktabs}
\usepackage{hyperref}
\usepackage{siunitx}

\usetheme{HM} % custom HM design

% --- Meta ---
\title[Geschäftsprozessmanagement -- Organisation]{Geschäftsprozessmanagement}
\subtitle{Organisation}
\author{Prof. Dr. Marcus Fischer}
\institute{Hochschule München University of Applied Sciences\\Fakultät für Mathematik und Informatik}
\date{Sommersemester 2025}

% Optional: Feinjustierung des linken Infoblocks auf der Titelfolie
\sethmleftinfo{Hochschule\\München\\University of\\Applied Sciences\\[1.4em]Fakultät für Mathematik und Informatik\\Professur für Enterprise Architecture Management und betriebliche Informationssysteme}

% If your logo has a different filename, set it here:
% \sethmlogo{hm_logo.pdf}

\begin{document}

\begin{frame}[plain]
  \titlepage
\end{frame}

\begin{frame}{Agenda}
  \hmframehighlight{Überblick}
  \tableofcontents
\end{frame}

\section{Einführung}
\begin{frame}{Geschäftsprozessmanagement}{Organisatorisches}
  \hmframehighlight{Die Prüfung}
  \begin{itemize}
    \item Die Prüfung ist schriftlich und dauert \SI{90}{\minute}.
    \item Alle Inhalte aus Vorlesung, Übung und Praxisprojekten sind prüfungsrelevant.
    \item Insgesamt können \num{90} Punkte erreicht werden.
    \item In der letzten Vorlesungswoche findet eine Probeklausur mit Besprechung statt.
  \end{itemize}
\end{frame}

\section{Beispielblöcke}
\begin{frame}{Werkzeuge}{Teamarbeit}
  \hmframehighlight{Methodische Unterstützung}
  \begin{block}{Hinweis}
    Kombinieren Sie digitale Whiteboards mit kurzen Abstimmungen, um Entscheidungen zu sichern.
  \end{block}
  \begin{exampleblock}{Best Practice}
    Wöchentliche Retrospektiven schaffen Raum für kontinuierliche Verbesserungen.
  \end{exampleblock}
\end{frame}

\section{Zusammenfassung}
\begin{frame}{Ausblick}{Nächste Schritte}
  \hmframehighlight{To-do-Liste}
  \begin{enumerate}
    \item Projektrollen final abstimmen und Dokumentation aktualisieren.
    \item Kommunikationstools vereinheitlichen und Zugänge bereitstellen.
    \item Vorbereitungsaufgaben bis zur kommenden Sitzung verteilen.
  \end{enumerate}
\end{frame}

\end{document}
